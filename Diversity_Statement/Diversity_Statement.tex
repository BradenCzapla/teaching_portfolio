\pagestyle{plain}
\chapter[Diversity Statement][Diversity Statement]{Diversity Statement}  \setstretch{1}

\begin{itemize}
\item Introduction
	\begin{itemize}
	\item Snapshot of how you embrace diversity.
	\item Share values and principles
	\item Demonstrate knowledge of institution's values
	\item Names experience and may include identity
	\end{itemize}
\item 3 narratives of how you applied
\begin{itemize}
	\item Can be teaching, research, or service (community, university, department)
	\item Embodied diversity, scholarly diversity, curricular diversity
	\item Each narrative must:
		\begin{itemize}
		\item Provide context
		\item Illustrate motivations
		\item Show active effort to think/work/engage with diversity
		\item Evidences success or continued work
		\end{itemize}
	\end{itemize}
\item Conclusion
\begin{itemize}
	\item Reinforce intro
	\item Reaffirm alignment with institution
	\end{itemize}
\end{itemize}

Experiences to draw from:
\begin{itemize}
\item Queer
\item Economic status
	\begin{itemize}
	\item Having grown up floating around the poverty line in a much more affluent area, I can understand some of the ways economic issues creep into the classroom. It can really be a black cloud that envelops much of your life. I am now conscious of assuming anyone has the means to do things I think are cheap. I prefer open source softwares and free resources instead of expensive software and texts.
	\end{itemize}
\item Living in NYC?
\end{itemize}



\begin{itemize}
\item Narrative 1
\item Narrative 2 - Research - Use of open source tools. Publishing them. Open science in general? ArXiV
\item Narrative 3 - Service? MEGA - keep costs down to be inclusive?
\end{itemize}