\pagestyle{plain}
\chapter[Professional Development][Professional Development]{Professional Development}
\raggedright

\textbf{Lead Teaching Fellowship}

{\color{red}The Lead Teaching Fellowship (LTF) program is a professional development opportunity for doctoral candidates interested in promoting furthering their pedagogical development. LTFs attend workshops and sponsor two teaching-related events in their home departments. My two workshops focused on active learning techniques that teaching assistants can integrate into their recitations plans and the ways in which mentoring students in research lab environments is a form of teaching.

\medskip Senior Lead Teaching Fellows (SLTFs) oversee and mentor a group of five LTFs). The LTFs I oversaw were paired with me due to our common interest in changing cultures in teaching. SLTFs also continue to develop their own knowledge of pedagogy by creating a learning community: three workshops devoted to a single topic. The focus on my learning community is metacognition, thinking about one's own thinking. The three workshops cover (1) literature surrounding the benefits of metacognitive practices in the classroom, (2) in-class techniques and assignments which promote metacognitive thinking, and (3) ways of projecting our thoughts on teaching into our web presence and teaching statements.}

For more information on the Lead Teaching Fellowship, follow the link \href{https://ctl.columbia.edu/graduate-instructors/opportunities-for-graduate-students/lead-teaching-fellows/}{\underline{here}}.

\medskip
\textbf{Innovative Teaching Summer Institute}

For more information on the Innovative Teaching Summer Institute, follow the link \href{https://ctl.columbia.edu/innovative-teaching-summer-institute/}{\underline{here}}.

\medskip
\textbf{Teaching Development Program}

For more information on the Teaching Development Program, follow the link \href{https://ctl.columbia.edu/graduate-instructors/ctl-teaching-development-program/}{\underline{here}}.

\medskip
\textbf{Pedagogy Workshop Participation}

{\color{red}Participation in myriad pedagogy-focused workshops, hosted by Columbia's Center for Teaching \& Learning and the School of Engineering and Applied Science, totalling over 100 hours of training. Topics covered include active learning techniques, backwards design of assignments, engaging with international students, inclusivity in teaching, metacognition, menetorship as teaching, and the scholarship of teaching, to name a few.}