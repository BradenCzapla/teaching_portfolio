\pagestyle{empty} % No page number
\begin{center}
\textbf{Methodology}
\end{center}

\singlespacing
The survey was composed of 14 sections: program information, doctoral-specific questions, masters-specific questions, alumni-specific questions, teaching preparation, academics, academic advising, social life, career services, department website, facilities, MEGA events, demographics, and free response. These topics were chosen specifically because they are areas in which MEGA either enact change itself, or reach out to other entities to ask them to enact change.

Based on information collected in the program information section, respondents were initially routed to one of three possible sections: doctoral, masters, or alumni-specific questions. From there, all masters students were sent to the academics sections. Any doctoral student who indicated that they had held a teaching assistant position in the department was then brought to the teaching preparation section. They were then sent to the academics section. Alumni were given the option to answer quality of life questions. If they answered yes, they were sent to the academics section. Otherwise, they were sent to the end. Once respondents got to the academics section, they progressed through the survey sequentially.

The survey was composed of three different question types: single-response categorical questions, multiple-response categorical questions, and single-response rating questions. All single-response rating questions were scored on a scale from one to five. One indicated a poor rating and five indicated an exceptional rating (e.g., totally dissatisfied to totally satisfied). Some single-response ratings questions also included a ``not applicable" option.   